\documentclass[letterpaper, 12pt]{article}

\usepackage[utf8]{inputenc}
\usepackage[T1]{fontenc}
\usepackage{color}
\usepackage[document]{ragged2e}
\usepackage{lmodern}
\usepackage[margin=1in]{geometry}
\usepackage{amsmath,amsfonts,amssymb,amsthm}
\usepackage{mathtools}
\usepackage{graphicx}
\usepackage[shortlabels]{enumitem}
\usepackage{tikz}
\usepackage{verbatim}
\usepackage{multirow}
\usetikzlibrary{arrows,shapes}
\usepackage[%
	pdftitle={TITLE},%
	hidelinks,%
]{hyperref}

%make macro for coloured text
\definecolor{red}{RGB}{210,0,0}
\definecolor{blue}{RGB}{0,0,170}
\newcommand{\red}[1]{{\color{red}{#1}}}
\newcommand{\blue}[1]{{\color{blue}{#1}}}
\newcommand{\green}[1]{{\color{green}{#1}}}
\newcommand{\yellow}[1]{{\color{yellow}{#1}}}
\newcommand{\purple}[1]{{\color{purple}{#1}}}
\newcommand{\white}[1]{{\color{white}{#1}}}

%override default second layer itemize to circle
\renewcommand{\labelitemii}{$\circ$}

\begin{document}

    
    \clearpage
    \vspace*{\fill}
    \begin{center}
        \begin{minipage}{\textwidth} 
            \title{CS241 Notes}
            \author{Jacky Zhao}
            \date{\today}
            \maketitle
        \end{minipage} 
    \end{center}
    \vfill
    \thispagestyle{empty}
    \newpage
    \setcounter{page}{1}

    \section{Module 1: Data Representation}
    Recall that integers are the underlying data of char. If we have:\\
    
    unsigned char a = 88;\\

    Printing a as a char will show "X", while printing a as an int will show "88".\\
    \bigskip

    If we write a into the file and open it with a text editor, it shows "X". This is because
    text editors interprets the ASCII value of the content of the file.\\
    If we open the same file with xxd, which gives raw value, it shows "01001000", the binary representation of 88.\\

    

\end{document}