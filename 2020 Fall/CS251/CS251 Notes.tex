\documentclass[letterpaper, 12pt]{article}

\usepackage[utf8]{inputenc}
\usepackage[T1]{fontenc}
\usepackage{color}
\usepackage[document]{ragged2e}
\usepackage{lmodern}
\usepackage[margin=1in]{geometry}
\usepackage{amsmath,amsfonts,amssymb,amsthm}
\usepackage{mathtools}
\usepackage{graphicx}
\usepackage[shortlabels]{enumitem}
\usepackage{tikz}
\usepackage{verbatim}
\usepackage{multirow}
\usetikzlibrary{arrows,shapes}
\usepackage[%
	pdftitle={TITLE},%
	hidelinks,%
]{hyperref}

%make macro for coloured text
\definecolor{red}{RGB}{210,0,0}
\definecolor{blue}{RGB}{0,0,170}
\newcommand{\red}[1]{{\color{red}{#1}}}
\newcommand{\blue}[1]{{\color{blue}{#1}}}
\newcommand{\green}[1]{{\color{green}{#1}}}
\newcommand{\yellow}[1]{{\color{yellow}{#1}}}
\newcommand{\purple}[1]{{\color{purple}{#1}}}
\newcommand{\white}[1]{{\color{white}{#1}}}

%override default second layer itemize to circle
\renewcommand{\labelitemii}{$\circ$}

\begin{document}

    
    \clearpage
    \vspace*{\fill}
    \begin{center}
        \begin{minipage}{\textwidth} 
            \title{CS251 Notes}
            \author{Jacky Zhao}
            \date{\today}
            \maketitle
        \end{minipage} 
    \end{center}
    \vfill
    \thispagestyle{empty}
    \newpage
    \setcounter{page}{1}

    \section{ARM Overview}
    What is ARM?\\
    \begin{itemize}
        \item Advanced RISC Machines
        \item RISC: Reduced Instruction Set Computing
    \end{itemize}
    We will be learning 64-bit ARM in this course, the textbook teaches ARM (Legv8).\\
    \bigskip
    \subsection{Registers}
    There will be a total of 32 registers. They can be used like variables in a program, but via ARM instruction.\\
    \begin{itemize}
        \item each register has 64 bits, or 8 bytes
        \item X0, X1, ... , X31
        \item X31 (XZR) always contains 0
        \item registers are hardware storage containers
    \end{itemize}
    \bigskip
    For example, suppose we have a high level code:\\
    $$f = (g+h)-(i+j)$$
    What would be the equivalent in ARM assembly?\\
    Assumption: the values of the variables are pre-looaded into the registers.\\
    \bigskip
    \begin{tabular}{ccccc}
        X1: f & X2: g & X3: h & X4: i & X5: j\\
    \end{tabular}
    \bigskip
    Here are the ARM instructions:\\
    ADD X6, X2, X3 ;;X6 holds a temp value\\
    ADD X7, X4, X5 ;;X7 holds a temp value\\
    SUB X1, X6, X7 ;;X1 has the final value\\
    \bigskip
    ADD, SUB instructions are known as \red{R-format instructions}\\

    \bigskip
    We are still imcomplete as
    \begin{itemize}
        \item we have not store the result into Data Memory (variable f)
        \item we have not show how to load variables into the registers
    \end{itemize}

    \bigskip
    

\end{document}