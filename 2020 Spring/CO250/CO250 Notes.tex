\documentclass[letterpaper, 12pt]{article}

\usepackage[utf8]{inputenc}
\usepackage[T1]{fontenc}
\usepackage{color}
\usepackage[document]{ragged2e}
\usepackage{lmodern}
\usepackage[margin=1in]{geometry}
\usepackage{amsmath,amsfonts,amssymb,amsthm}
\usepackage{mathtools}
\usepackage{graphicx}
\usepackage[shortlabels]{enumitem}
\usepackage{tikz}
\usepackage[%
	pdftitle={CO250 Notes},%
	hidelinks,%
]{hyperref}

\begin{document}

    
    \clearpage
    \vspace*{\fill}
    \begin{center}
        \begin{minipage}{\textwidth} 
            \title{CO250 Spring 2020}
            \author{Jacky Zhao}
            \date{\today}
            \maketitle
        \end{minipage} 
    \end{center}
    \vfill
    \thispagestyle{empty}
    \newpage
    \setcounter{page}{1}

    \section{Introduction}
    \subsection{Abstract Optimization Problem}
    An \textit{abstract optimization problem (P)} is of the following form:
    \begin{itemize}
        \item {\color{red}Given}: a set $\textbf{A} \subseteq \mathbb{R}^n$ and a function $f:\textbf{A} \rightarrow \mathbb{R}$
        \item {\color{red}Goal}: find $x \in \textbf{A}$ that minimizes/maximizes $f$
        \item {\color{red}Bad news}: Hard to solve \& may not be well defined
    \end{itemize}
    
    \bigskip
    We look at 3 special cases of (P) in this course:
    \begin{enumerate}
        \item {\color{red}Linear Programming (LP)}
        \begin{itemize}
            \item $\textbf{A}$ is simply given by \textit{linear} constrains, and $f$ is a \textit{linear} function
        \end{itemize}
        \item {\color{red}Integer Programming (IP)}
        \begin{itemize}
            \item Same as above, but now we want max/min over the \textit{integer} points in $\textbf{A}$
        \end{itemize}
        \item {\color{red}Non-linear Programming (NLP)}
        \begin{itemize}
            \item $\textbf{A}$ is given by \textit{non-linear} constrains, and $f$ is a \textit{non-linear} function
        \end{itemize}
    \end{enumerate}
    
    \subsubsection{Example: Water Tech}
    WaterTech produces 4 products, $P = \{1, 2, 3, 4\}$, from the following resources:\\
    \begin{itemize}
        \item time on two machines
        \item skilled and unskilled labour
    \end{itemize}
    The following table gives precise requirements:\\

    \bigskip
    \begin{tabular}{|c|c|c|c|c|c|}
        \hline
        Product & Machine 1 & Machine 2 & Skilled Labour & Unskilled Labour & Unit Sale Price\\
        \hline
        1 & 11 & 4 & 8 & 7 & 300\\
        \hline
        2 & 7 & 6 & 5 & 8 & 260\\
        \hline
        3 & 6 & 5 & 5 & 7 & 220\\
        \hline
        4 & 5 & 4 & 6 & 4 & 180\\
        \hline
    \end{tabular}
    \bigskip
    
    {\color{red} Restrictions:}\\
    \begin{itemize}
        \item WaterTech has 700h on machine 1 and 500h on machine 2 available
        \item it can purchase 600h of skilled labour at \$8 per hour and at most 650h of unskilled labour at \$6 per hour
    \end{itemize}
    {\color{red} Question:}\\
    How much of each product should WaterTech produce in order to maximize profit?\\
    \pagebreak
    \subsection{Ingredients of a math model:}
    \begin{itemize}
        \item {\color{red}Decision variables:} Capture unknown information
        \item {\color{red}Constraints:} Describe which assignments to variables are {\color{red}feasible}
        \item {\color{red}Objective function:} A function of the variables that we would like to maximize/minimize
    \end{itemize}

    \subsection{Variables}

    WaterTech needs to decide how many units of each product to produce, so introduce some variables:
    \begin{itemize}
        \item $x_i$ for number of labour to purchase
        \item $y_s$, $y_u$ for number of hours of skilled/unskilled labour to purchase
    \end{itemize}

    \subsection{Constrains}
    What makes an assignment to $\{x_i\} \in P, y_s, y_u$ a feasible assignment?\\
    \bigskip
    For example, a production plan described by an assignment may not use more than 700h of time on machine 1
    $$11x_1 + 7x_2 + 6x_3 + 5x_4 \leq 700$$

    Similarly, we may not use more than 500h of machine 2 time
    $$4x_1 + 6x_2 + 5x_3 + 4x_4 \leq 500$$

    Producing $x_i$ units of product $i \in P$ must require less than $y_s$ units of skilled labour
    $$8x_1 + 5x_2 + 5x_3 + 6x_4 \leq y_s$$

    Similar story for unskilled labour:
    $$7x_1 + 8x_2 + 7x_3 + 5x_4 \leq y_u$$

    Since amount of labour that can be purchased is limited, we also have
    $$y_s \leq 600$$
    $$y_u \leq 650$$

    \subsection{Objective Function}
    Revenue from sales:\\
    $$300x_1 + 260x_2 + 220x_3 + 180x_4$$

    Cost of labour:\\
    $$8y_s + 6y_u$$

    Objective Function:\\
    {\begin{center}
        maximize $300x_1 + 260x_2 + 220x_3 + 180x_4 - 8y_s - 6y_u$
    \end{center}
    \pagebreak
    
    {\large\textbf{The complete model for WaterTech problem is:}}\\
    \begin{center}
        \begin{tabular}{r l}
            max & $300x_1 + 260x_2 + 220x_3 + 180x_4 - 8y_s - 6y_u$\\
            s.t & $11x_1 + 7x_2 + 6x_3 + 5x_4 \leq 700$\\
            & $4x_1 + 6x_2 + 5x_3 + 4x_4 \leq 500$\\
            & $8x_1 + 5x_2 + 5x_3 + 6x_4 \leq y_s$\\
            & $7x_1 + 8x_2 + 7x_3 + 5x_4 \leq y_u$\\
            & $y_s \leq 600$\\
            & $y_u \leq 650$\\
            & $x_1, x_2, x_3, x_4, y_u, y_s \geq 0$\\
        \end{tabular}
    \end{center}

    Solution obtained via CPLEX is:\\
    \begin{align*}
        x &= (16 + \frac{2}{3}, 50, 0, 33 + \frac{1}{3})^T\\
        y_s &= 583 + \frac{1}{3}\\
        y_u &= 650\\
        Profit &= 15433 + \frac{1}{3}
    \end{align*}

    Notice that the solution is fractional, which may or may not be correct depending on the question\\
    \bigskip
    \subsection{Correctness of Model}
    First, define some terminologies:\\
    \begin{itemize}
        \item Word description of problem
        \begin{itemize}
            \item Similarly, a solution to the word description is an assignment to the unknowns\\
        \end{itemize}
        \item Formulation
        \begin{itemize}
            \item A solution to the formulation is an assignment to all of its variables\\
        \end{itemize}
    \end{itemize}
    
    A solution feasible if all constrains are satisfied, optimal if no other feasible solution exists\\
    \bigskip
    One way to show correctness is to define a mapping between feasible solutions to the word description, and feasible solutions to the model, and vice versa.\\
    \pagebreak

    \section{Linear Program Model (LP)}

    \subsection{Linear Functions}
    Affine Functions
    \begin{itemize}
        \item A function $f : \mathbb{R}^n \rightarrow \mathbb{R}$ is affine if $f(x) = \alpha^Tx + \beta$ for $\alpha \in \mathbb{R}^n, \beta \in \mathbb{R}$
    \end{itemize}
    Linear Functions
    \begin{itemize}
        \item An affine function with $\beta = 0$
    \end{itemize}

    \subsection{Linear Program}
    Linear Program
    \begin{itemize}
        \item the optimization problem $$max/min\{f(x):f_i(x) \leq b_i, \forall1 \leq i \leq m, x \in \mathbb{R}^n\}$$ is a linear program if $f$ is affine and $g_1,...,g_m$ is finite number of linear functions\\
    \end{itemize}
    
    \bigskip
    Some notes:
    \begin{itemize}
        \item dividing by variables is not allowed in LP
        \item can NOT have strict inequalities
        \item must have FINITE number of constraints
    \end{itemize}
    
    Example:\\
    \bigskip
    \begin{tabular}{rl}
        max & $\frac{-1}{x_1} - x_3$\\
        s.t. & $2x_1 + x_3 < 3$\\
        &$x_1 + \alpha x_2 = 2\hspace{1cm}\forall\alpha\in\mathbb{R}$\\
    \end{tabular}
    \bigskip

    Going back to the WaterTech problem, the model we created was in fact a linear program!\\

    \subsection{LP Models: Multiperiod Models}
    A multiperiod model is a problem where:\\
    \begin{itemize}
        \item time is split into periods
        \item we have to make a decision in each period
        \item all decisions influences the final outcome
    \end{itemize}
    Example:\\
    KW Oil is a local supplier of heating oil, it needs to decide how much oil to purchase in order to satisfy demand of its customers.\\
    \bigskip
    \begin{tabular}{|c|c|c|c|c|}
        \hline
        Month & 1 & 2 & 3 & 4\\
        \hline
        Demand($l$) & 5000 & 8000 & 9000 & 6000\\
        \hline
        Price(\$/$l$) & 0.75 & 0.72 & 0.92 & 0.90\\
        \hline
    \end{tabular}
    \bigskip

    Question: When should we purchase how much oil when the goal is to min overall total cost?\\
    Additional Complication: The company has a storage tank that\\
    \begin{itemize}
        \item has a capacity of 4000 litres of oil
        \item currently (beginning of month 1) contains 2000 litres of oil
    \end{itemize}
    Assumption: Oil is delivered at the beginning of the month, and consumption occurs int he middle of the month

    \bigskip
    {\large\textbf{Variables}}
    \begin{itemize}
        \item Need to decide how many litres of oil to purchase in each month $i$
        \begin{itemize}
            \item make variable $p_i$ for $i \in [4]$
        \end{itemize}
        \item How much oil is stored in the tank at the beginning of month $i$?
        \begin{itemize}
            \item make variable $t_i$ for $i \in [4]$
        \end{itemize}
    \end{itemize}

    {\large\textbf{Objective Function}}\\
    Minimize cost of oil purchased
    $$min\hspace{1cm} 0.75p_1 + 0.72p_2 + 0.92p_3 + 0.90p_4$$

    {\large\textbf{Constrains}}\\
    We need 
    \begin{center}
        $p_i + t_i \geq$ (demand in month $i$)\\
    \end{center}
    Balancing equation we get
    \begin{center}
        $p_i + t_i = $ (demand in month $i$) $ + t_{i+1}$\\
    \end{center}

    So we have the following four constrains
    \begin{align*}
        p_1 + 2000 & = 5000 + t_2\\
        p_2 + t2 & =  5000 + t_3\\
        p_3 + t3 & =  5000 + t_4\\
        p_4 + t4 & \geq  6000
    \end{align*}

    {\large\textbf{Complete LP for KW Oil}}\\
    \bigskip
    \begin{tabular}{rl}
        min & $0.75p_1 + 0.72p_2 + 0.92p_3 + 0.90p_4$\\
        s.t. & {$\begin{aligned}[t]
            p_1 + 2000 & = 5000 + t_2\\
            p_2 + t2 & =  5000 + t_3\\
            p_3 + t3 & =  5000 + t_4\\
            p_4 + t4 & \geq  6000\\
            t_1 &= 2000\\
            t_i &\leq 4000 \hspace{1cm} (i=2, 3, 4)\\
            t_1, p_i &\geq 0 \hspace{1cm} (i=1, 2, 3, 4) \end{aligned}$}
    \end{tabular}
    \bigskip
    
    Solving the LP gives the solution:\\
    $p = (3000,12000,5000,6000)^T$\\
    $t = (2000, 0, 4000, 0)^T$

    \pagebreak
    \section{Integer Program (IP)}
    Recall the WaterTech problem\\
    \begin{center}
        \begin{tabular}{r l}
            max & $300x_1 + 260x_2 + 220x_3 + 180x_4 - 8y_s - 6y_u$\\
            s.t & $11x_1 + 7x_2 + 6x_3 + 5x_4 \leq 700$\\
            & $4x_1 + 6x_2 + 5x_3 + 4x_4 \leq 500$\\
            & $8x_1 + 5x_2 + 5x_3 + 6x_4 \leq y_s$\\
            & $7x_1 + 8x_2 + 7x_3 + 5x_4 \leq y_u$\\
            & $y_s \leq 600$\\
            & $y_u \leq 650$\\
            & $x_1, x_2, x_3, x_4, y_u, y_s \geq 0$\\
        \end{tabular}
    \end{center}

    \begin{align*}
        x &= (16 + \frac{2}{3}, 50, 0, 33 + \frac{1}{3})^T\\
        y_s &= 583 + \frac{1}{3}\\
        y_u &= 650\\
        Profit &= 15433 + \frac{1}{3}
    \end{align*}

    Fractional solutions are often not desirable! Can we force the solution to be integer?\\
    \bigskip
    {\large\textbf{Integer Program}}\\
    \begin{itemize}
        \item an integer program is a linear program with added integrality constraints for some/all the variables
        \item we call an IP mixed if there are integer and fractional variables, and pure otherwise
        \item the difference between LPs and IPs is subtle, but LPs are easy to solve, IPs are not!
    \end{itemize}
    Integer program is provably difficult to solve!\\
    \begin{itemize}
        \item An algorithm is efficient if its running can be bounded by a polynomial of the input size of the instance\\
        \item LPs can be solved efficiently
        \item IPs are very unlikely to have efficient algorithms!
    \end{itemize}

    \subsection{IP Models: Knapsack}
    Example:\\
    KitchTech Shipping is a company wishes to ship crates from Toronto to Kitchener.
    Each crate type has a weight and value, and the total weight of crates shipped must not exceed 10,000 lbs.\\
    Goal: Maximize the total value of shipped goods.\\
    \bigskip
    \begin{center}
        \begin{tabular}{|c|c|c|c|c|c|c|}
            \hline
            Type & 1 & 2 & 3 & 4 & 5 & 6\\
            \hline
            weight (lbs) & 30 & 20 & 30 & 90 & 30 & 70\\
            \hline
            value (\$) & 60 & 70 & 40 & 70 & 20 & 90\\
            \hline
        \end{tabular}
    \end{center}

    \bigskip
    \textbf{Variables:}\\
    One variable $x_i$ for the number of crates of type $i$ to pack.\\
    \bigskip
    \textbf{Constraints:}\\
    The total weight of crates picked must not exceed 10000 lbs.\\
    $$30x_1 + 20x_2 + 30x_3 + 90x_4 + 30x_5 + 70x_6 \leq 10000$$
    \textbf{Objective function:}\\
    Maximize the total value\\
    $$max \hspace{1cm} 60x_1 + 70x_2 + 40x_3 + 70x_4 + 20x_5 + 90x_6$$

    \textbf{Complete IP model for KitchTech Shipping:}\\
    \begin{center}
        \begin{tabular}{rl}
            max & $60x_1 + 70x_2 + 40x_3 + 70x_4 + 20x_5 + 90x_6$\\
            s.t. & $30x_1 + 20x_2 + 30x_3 + 90x_4 + 30x_5 + 70x_6 \leq 10000$\\
            & $x_i \geq 0 \hspace{1cm} (i\in[6])$\\
            & $x_i$ integer $(i \in [6])$\\
        \end{tabular} 
    \end{center}
    
    Let's make this shit more complicated with more rules...\\
    Suppose that:
    \begin{enumerate}
        \item we must not send more than 10 crates of the same type
        \item we can only send crates of type 3, if we send at least 1 crate of type 4
    \end{enumerate}}
    Note that we can send at most 10 crates of type 3 by the previous constraints!\\
    By adding the following constraint, the added requirements is fulfilled:
    $$ x_3 \leq 10x_4$$
    \textbf{proving correctness of the added constraint:}
    \begin{itemize}
        \item $x_4 \geq 1 \rightarrow$  new constraint is redundant
        \item $x_4 = 0 \rightarrow$  new constraint becomes $x_3 \leq 0$
    \end{itemize}

    Suppose we add another rule where we must:\\
    \begin{enumerate}
        \item take a total of at least 4 crates of type 1 or 2, or
        \item take at least 4 crates of type 5 or 6
    \end{enumerate}

    \textbf{strategy:}\\
    Create a new variable y such that:
    \begin{itemize}
        \item $y = 1 \rightarrow x_1 + x_2 \geq 4$
        \item $y = 0 \rightarrow x_5 + x_6 \geq 4$
        \item and force $y$ to take on value 0 or 1
    \end{itemize}

    So we add the following constraints:\\
    \begin{itemize}
        \item $x_1 + x_2 \geq 4y$
        \item $x_5 + x_6 \geq 4(1 - y)$
        \item $0 \leq y \leq 1$
        \item y integer
    \end{itemize}

    The variable $y$ we added is called a binary variable. These are very useful for modelling logical constraints of the form:\\
    \begin{itemize}
        \item Condition (A or B) and C $\rightarrow$ D
    \end{itemize}

    So the finalized model would be:\\
    \begin{center}
        \begin{tabular}{rl}
            max & $60x_1 + 70x_2 + 40x_3 + 70x_4 + 20x_5 + 90x_6$\\
            s.t. & $30x_1 + 20x_2 + 30x_3 + 90x_4 + 30x_5 + 70x_6 \leq 10000$\\
            & $ x_3 \leq 10x_4$\\
            & $x_1 + x_2 \geq 4y$\\
            & $x_5 + x_6 \geq 4(1 - y)$\\
            & $x_i \geq 0 \hspace{1cm} (i\in[6])$\\
            & $0 \leq y \leq 1$\\
            & y integer\\
            & $x_i$ integer $(i \in [6])$\\
        \end{tabular} 
    \end{center}

    \subsection{IP Models: Scheduling}
    Example:\\
    The neighborhood coffee shop is open on workdays. The daily demand for workers is given in the table.
    Each worker works for 4 consecutive days and has one day off.\\
    Goal: Hire the smallest number of workers so that the demand can be met\\
    \bigskip
    \begin{tabular}{|c|c|c|c|c|}
        \hline
        Mon & Tues & Wed & Thurs & Fri\\
        \hline
        3 & 5 & 9 & 2 & 7\\
        \hline
    \end{tabular}
    
    \bigskip
    \textbf{Variables:}\\
    Introduce variable $x_d$ for every $d \in \{M,T,W,Th,F\}$ counting the number of people to hire with starting day $d$\\
    \bigskip
    \textbf{Objective function:}\\
    Minimize the total number of people hired:\\
    $$min \hspace{1cm} x_M + x_T + x_W + x_{Th} + x_F$$
    \textbf{Constraints:}\\
    We need to ensure that enough people work on each of the days.\\
    \textbf{Question:} Given a solution, how many people work on Monday?\\
    \textbf{Answer:} All but those that start on Tuesday, i.e.
    $$x_M + x_W + x_{Th} + x_F$$
    And it must be greater than or equal to the number of workers required\\
    So the complete LP is:\\
    \bigskip
    \begin{center}
        \begin{tabular}{rl}
            min & $x_M + x_T + x_W + x_{Th} + x_F$\\
            s.t. & $x_M + x_W + x_{Th} + x_F \geq 3$\\
            & $x_M + x_T + x_{Th} + x_F \geq 5$\\
            & $x_M + x_T + x_W + x_F \geq 9$\\
            & $x_M + x_T + x_W + x_{Th} \geq 2$\\
            &  $x_T + x_W + x_{Th} + x_F \geq 7$\\
            & $x \geq 0, x$ integer\\
        \end{tabular}
    \end{center}
    
    \pagebreak
    





\end{document}