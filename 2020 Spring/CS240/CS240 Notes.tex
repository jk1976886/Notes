\documentclass[letterpaper, 12pt]{article}

\usepackage[utf8]{inputenc}
\usepackage[T1]{fontenc}
\usepackage{color}
\usepackage[document]{ragged2e}
\usepackage{lmodern}
\usepackage[margin=1in]{geometry}
\usepackage{amsmath,amsfonts,amssymb,amsthm}
\usepackage{mathtools}
\usepackage{graphicx}
\usepackage[shortlabels]{enumitem}
\usepackage{tikz}
\usepackage{verbatim}
\usepackage{multirow}
\usetikzlibrary{arrows,shapes}
\usepackage[%
	pdftitle={CS240 Notes},%
	hidelinks,%
]{hyperref}

%make macro for coloured text
\definecolor{red}{RGB}{210,0,0}
\definecolor{blue}{RGB}{0,0,170}
\newcommand{\red}[1]{{\color{red}{#1}}}
\newcommand{\blue}[1]{{\color{blue}{#1}}}
\newcommand{\green}[1]{{\color{green}{#1}}}
\newcommand{\yellow}[1]{{\color{yellow}{#1}}}
\newcommand{\purple}[1]{{\color{purple}{#1}}}
\newcommand{\white}[1]{{\color{white}{#1}}}

%override default second layer itemize to circle
\renewcommand{\labelitemii}{$\circ$}

\begin{document}

    
    \clearpage
    \vspace*{\fill}
    \begin{center}
        \begin{minipage}{\textwidth} 
            \title{CS240 Notes}
            \author{Jacky Zhao}
            \date{\today}
            \maketitle
        \end{minipage} 
    \end{center}
    \vfill
    \thispagestyle{empty}
    \newpage
    \setcounter{page}{1}

    \section{Course Objectives}
    \subsection{Overview}
    What is this course about?
    \begin{itemize}
        \item When first learning to program, we emphasize \red{correctness}
        \item Starting with this course, we will also be converned with \red{efficiency}
        \item We will study efficient methods of \red{storing, accessing, and performing operations} on large collections of data.
        \item Typical operations include: \red{inserting} new data items, \red{deleting} data items, \red{searching} for specific data items, \red{sorting}\\
        \item We will consider various \red{abstract data types} (ADTs) and how to implemnet them efficiently using appropriate \red{data structures}.
        \item There is a strong emphasis on mathematical analysis in the course
        \item Algorithms are presented using pseudocode and analyzed using order notation (big-O, etc.)
    \end{itemize}
    \bigskip
    \textbf{\red{Course Topics}}:
    \begin{itemize}
        \item big-O analysis
        \item priority queues and heaps
        \item sorting, selection
        \item binary search trees, AVL trees, B-trees
        \item skip lists
        \item hashing
        \item quadtrees, kd-trees
        \item range search
        \item tries
        \item string matching
        \item data compression
    \end{itemize}
    \pagebreak
    \textbf{Required knowledge:}
    \begin{itemize}
        \item arrays, linked lists (3.2- 3.4)
        \item strings (3.6)
        \item stacks, queues (4.2 - 4.6)
        \item abstract data types (4 - intro, 4.1, 4.8 - 4.9)
        \item recursie algorithms (5.1)
        \item binary trees (5.4 - 5.7)
        \item sorting (6.1 - 6.4)
        \item binary search (12.4)
        \item binary search trees (12.5)
        \item probability and expectations
    \end{itemize}
    \pagebreak
    \subsection{General Terminologies}
    The core of CS240 is:\\
    \begin{center}
        Given problem $\Pi$, design algorithm $A$ that solves it, and analyze its \red{efficiency}
    \end{center}
    So what is a problem, an algorithms, and how do you quantify efficiency?\\
    \bigskip
    \red{Problem}
    \begin{itemize}
        \item Given a \red{problem instance}, carry out a particular computational task
        \item Ex. Sorting is a problem
    \end{itemize}
    \red{Problem Instance}
    \begin{itemize}
        \item \red{Input} for the specified problem
    \end{itemize}
    \red{Problem Solution}
    \begin{itemize}
        \item \red{Output} (correct answer) for the specified problem instance
    \end{itemize}
    \red{Size of a problem instance}
    \begin{itemize}
        \item \red{$Size(I)$} is a positive integer which is a measure of the size of the instance $I$
    \end{itemize}

    \bigskip

    \red{Algorithm}
    \begin{itemize}
        \item a \red{step-by-step process} (e.g. described in pseudocode) for carrying out a series of computations,
        given an arbitrary problem instance $I$
    \end{itemize}
    \red{Algorithm solving a problem}
    \begin{itemize}
        \item an algorithm $A$ \red{solves} a problem $\Pi$ if, for every instance $I$ of $\Pi$, $A$ finds
        (computes) a valid solution for the instance $I$ in finite time
    \end{itemize}
    \red{Program}
    \begin{itemize}
        \item an \red{implementation} of an algorithm using a specified computer language
    \end{itemize}

    \red{Pseudocode}
    \begin{itemize}
        \item a method of communicating an algorithm to another person
        \item in contrast, a program is a method of communicating an algorithm to a computer
        \item General rules of pseudocode:
        \begin{itemize}
            \item omits obvious details (variable declarations)
            \item has limited, if any, error detection
            \item sometimes uses English descriptions
            \item sometimes usus mathematical notation
        \end{itemize}
    \end{itemize}
    \pagebreak
    \subsection{Algorithms and programs}
    For a problem $\Pi$, we can have several algorithms.\\
    For an algorithm $A$ solving $\Pi$, we can have several programs (implementations)\\
    \bigskip
    Algorithms in practice: Given a problem $\Pi$:\\
    \begin{enumerate}
        \item \textbf{\red{Algorithm Design:}} Design an algorithm $A$ that solves $\Pi$
        \item \textbf{\red{Algorithm Analysis:}} Assess \red{correctness} and \red{efficiency} of $A$
        \item If acceptable (correct and efficient), implement $A$.
    \end{enumerate}
    


\end{document}