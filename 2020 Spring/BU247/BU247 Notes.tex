\documentclass[letterpaper, 12pt]{article}

\usepackage[utf8]{inputenc}
\usepackage[T1]{fontenc}
\usepackage{color}
\usepackage[document]{ragged2e}
\usepackage{lmodern}
\usepackage[margin=1in]{geometry}
\usepackage{amsmath,amsfonts,amssymb,amsthm}
\usepackage{mathtools}
\usepackage{graphicx}
\usepackage[shortlabels]{enumitem}
\usepackage{tikz}
\usepackage{verbatim}
\usepackage{multirow}
\usetikzlibrary{arrows,shapes}
\usepackage[%
	pdftitle={BU247 Notes},%
	hidelinks,%
]{hyperref}

%make macro for coloured text
\definecolor{red}{RGB}{210,0,0}
\definecolor{blue}{RGB}{0,0,170}
\newcommand{\red}[1]{{\color{red}{#1}}}
\newcommand{\blue}[1]{{\color{blue}{#1}}}
\newcommand{\green}[1]{{\color{green}{#1}}}
\newcommand{\yellow}[1]{{\color{yellow}{#1}}}
\newcommand{\purple}[1]{{\color{purple}{#1}}}
\newcommand{\white}[1]{{\color{white}{#1}}}

%override default second layer itemize to circle
\renewcommand{\labelitemii}{$\circ$}

\begin{document}
    \clearpage
    \vspace*{\fill}
    \begin{center}
        \begin{minipage}{\textwidth} 
            \title{BU247 Notes}
            \author{Jacky Zhao}
            \date{\today}
            \maketitle
        \end{minipage} 
    \end{center}
    \vfill
    \thispagestyle{empty}
    \newpage
    \setcounter{page}{1}
    \pagebreak

    \section{Chapter 2}
    How management accounting supports internal decision making:
    \begin{itemize}
        \item Pricing
        \begin{itemize}
            \item Facing a market-determined price, an organization will use product cost information to decide whether its cost structure will allow it to compete profitably
            \item When the organization can set the price, they often set it so that it is an increment of its product cost - aka \red{cost plus pricing}
        \end{itemize}
        \item Product Planning
        \begin{itemize}
            \item Organizations use \red{target costing} to focus efforts during product and process design on developing a product that has a good profit potential in view of market requirements
        \end{itemize}
        \item Budgeting
        \begin{itemize}
            \item Most widespread use of cost info is in budgeting, a management accounting tool that projects income and costs for various levels of production and sales activity
            \item Budgets provide the basis for earnings forecasts, which is to be issued to the stock market
        \end{itemize}
        \item Performance Evaluation
        \begin{itemize}
            \item Managers compare the actual results from the budget period with expectations that were reflected in the budget prepared to assess performance
        \end{itemize}
        \item Contracting
        \begin{itemize}
            \item In cost reimbursement contracts, organizations are reimbursed their cost plus a profit increment for the goods or services they provide under teh contract
            \item Governments are frequent and large-scale users of this type of contracts
        \end{itemize}
    \end{itemize}

    \pagebreak

    \red{Cost Objects}
    \begin{itemize}
        \item any item for which a cost is to be determined
        \item Examples:
        \begin{itemize}
            \item products, product lines, production lines, services, organization departments, and even the entire organizations
        \end{itemize}
    \end{itemize}

    \red{Direct Cost}
    \begin{itemize}
        \item the cost has a direct cause-and-effect relationship with the cost objects
        \item Defining Characteristic:
        \begin{itemize}
            \item the cost would not exist absent the cost object
        \end{itemize}
    \end{itemize}

    \red{Indirect Cost}
    \begin{itemize}
        \item no direct cause-and-effect relationship with the cost objects
        \item Example:
        \begin{itemize}
            \item weekly wage of car assembly workers (unrelated to \# of car produced)
            \item However, if we have unit wage of \$20/unit for workers, then this wage becomes direct cost
        \end{itemize}
    \end{itemize}

    \red{Variable Costs}
    \begin{itemize}
        \item costs that vary directly with some underlying level of activity , such as number of units produced
        \item \red{Direct material cost} is a form of variable cost
        \item Defining Characteristic:
        \begin{itemize}
            \item total variable cost depends on how much of the resource is used
            \item resources that are attributable is variable, such as wood used to make a chair cannot be used to make another
        \end{itemize}
        \item 
    \end{itemize}



\end{document}